\documentclass{article}
\usepackage{amsmath,amstext,amsfonts}
\newcommand{\NAT}{{\mathbb N}}
\newcommand{\SET}[1]{{\{{#1}\}}}
\newcommand{\COST}{{\textsf{cost}}}
\newcommand{\TO}{\rightarrow}
\newcommand{\ROOT}{{\textsf{root}}}
\newcommand{\MOTHER}{{\textsf{mother}}}
\newcommand{\DAUGHTERS}{{\textsf{daughters}}}
\newcommand{\DOWN}{{\textsf{down}}}
\newcommand{\EQDOWN}{{\textsf{eqdown}}}
\newcommand{\UNION}{\cup}
\newcommand{\TUP}[1]{{\langle{#1}\rangle}}
\newcommand{\EPS}{\epsilon}
\newcommand{\AT}[2]{@^{#1}_{#2}}
\newcommand{\CONVEX}{{\textsf{convex}}}
\newcommand{\IMPLIES}{\Rightarrow}
\title{Progamming Contest}
\setlength{\parskip}{\medskipamount}
\begin{document}
\maketitle

The basic idea, due to Andreas, is that the input document gives us an
upper bound on the number of elements $n_E$ that may occur in the
output document.

An input document consists of a finite set $C$ of data items.  These
data items are numbered by successive integers: we capture this by a
function $\pi:C\rightarrow\NAT$ which is a bijection between $C$ and
$\SET{1\ \ldots\ n_C}$ where $n_C=|C|$.  Furthermore, we assume given
a set $A$ of attributes and an assignment $\alpha_C:C\times A
\rightarrow \NAT$.

We assume given a set $T$ of tags and we define the cost of a tag by a
function $\COST_T:T\TO\NAT$:
\begin{center}
\begin{tabular}{c@{\hspace{1cm}}l@{}l}
\verb+<b> ... </b>+ & $\COST_T(\texttt{b})$ & ${}=7$\\
\verb+<pl> ... </pl>+ & $\COST_T(\texttt{pl})$ & ${}=9$
\end{tabular}
\end{center}
an upper bound for the number of element is:
\begin{gather*}
\lfloor{{\Sigma}\SET{\COST_T(e)\mid e\text{ in document}}/7}\rfloor
\end{gather*}
it may be possible to derive a tighter bound when e.g.\ the input
document contains redundant elements etc\ldots

Our formalization of the output document considers the set $C$ of data
items ($|C|=n_C$) and a set $E$ of elements ($|E|=n_E$).  We write
$V=E\uplus C$ for the set of nodes in the tree representation of the
document.  We also assume an additional node $\ROOT$ to serve as the
root node of the document.  We write $V^*=V\uplus\ROOT$.  Not all
elements in $E$ will necessarily be used (hopefully not): in our
formalization, these unused elements will be empty and immediate
daughters of the $\ROOT$.  We consider the following functions:
\begin{align*}
\MOTHER &: V\TO V^*\\
\DAUGHTERS &: V^*\TO 2^V\\
\DOWN &: V^*\TO 2^V\\
\EQDOWN &: V^*\TO 2^{V^*}
\end{align*}
The treeness condition is satisfied when $\forall w\in V^*$ and
$\forall w'\in V$:
\begin{align*}
\EQDOWN(w)&=\SET{w}\uplus\DOWN(w)\\
\DOWN(w)&={\UNION}\SET{\EQDOWN(w')\mid w'\in\DAUGHTERS(w)}\\
w=\MOTHER(w')\ &\equiv\ w'\in\DAUGHTERS(w)\\
V&={\uplus}\SET{\DAUGHTERS(w)\mid w\in V^*}
\end{align*}
Posing $V=\SET{w_1,\ldots,w_n}$,
equation $\DOWN(w)={\UNION}\SET{\EQDOWN(w')\mid w'\in\DAUGHTERS(w)}$
can be implemented by the \emph{selection union} constraint:
\begin{gather*}
\DOWN(w)={\UNION}\TUP{\EQDOWN(w_1),\ldots,\EQDOWN(w_n)}[\DAUGHTERS(w)]
\end{gather*}

We introduce an additional fictional tag $\EPS$ to be assigned to an
element which is not used.  We write $T^*=T\uplus\SET{\EPS}$.  We
introduce the function $\tau:E\TO T^*$ to represent an assignment of
tags to elements.

Each attribute $a\in A$ has a default value $\delta(a)$.  We introduce
the function $\alpha:V^*\times A\TO\NAT$ to represent an assignment of
attribute values to nodes.
\begin{gather*}
\alpha(w,a) = \left\{
\begin{array}{l@{\qquad}l}
\alpha_C(w,a) & \text{for } w\in C\\
\delta(a) & \text{for } w=\ROOT\\
\AT{\tau(w)}{a}(\alpha(\MOTHER(w),a)) & \text{for } w\in E
\end{array}
\right.
\end{gather*}
where $\AT{t}{a}:\NAT\TO\NAT$ is the function which computes the value
of attribute $a$ inside a tag $t$ given its value outside.  For most
attributes it is the identity function.

There are two difficulties in the last equation: (1) computing
$\alpha$ at the unknown point $(\MOTHER(w),a)$ and (2) computing
$\AT{\tau(w)}{a}$ at the unknown point $\alpha(\MOTHER(w),a)$.  This
can be achieved using the following reduction:
\begin{align*}
\alpha(w',a) &= \TUP{\alpha(w_1,a),\ldots,\alpha(w_n,a)}[w']\\
\AT{t}{a}(x) &= \TUP{\AT{t}{a}(1),\ldots,\AT{t}{a}(k_a)}[x]
\end{align*}
where we assume that the possible values for attribute $a$ range from
$1$ to $k_a$.

We write $S$ for the data range $\SET{1,\ldots,n_C}$ and introduce the
scope function $\sigma:V^*\TO 2^S$.  For all $w\in V^*$, its scope is
an interval:
\begin{gather*}
\CONVEX(\sigma(w))
\end{gather*}
furthermore, it can be computed inductively:
\begin{gather*}
\sigma(w) = \left\{
\begin{array}{l@{\hspace{1cm}}l}
S & \text{if } w=\ROOT\\
\SET{\pi(w)} & \text{if } w\in C\\
{\UNION}\SET{\sigma(w')\mid w'\in\DAUGHTERS(w)} & \text{if } w\in E
\end{array}
\right.
\end{gather*}
An element $w\in E$ is unused iff it is empty.  Furthermore, in that
case it is an immediate daughter of the $\ROOT$.
\begin{align*}
\tau(w)&=\EPS\ &\equiv\ |\sigma(w)|=0\\
\tau(w)&=\EPS\ &\IMPLIES\ \ROOT=\MOTHER(w)
\end{align*}
Finally, we introduce a cost function $\kappa:E\TO\NAT$:
\begin{gather*}
\kappa(w) = \TUP{\COST_T(t_1),\ldots,\COST_T(t_k)}[\tau(w)]
\end{gather*}
where we pose $T^*=\SET{t_1,\ldots,t_k}$ and $\COST_T(\EPS)=0$.
The goal is to minimize the total cost:
\begin{gather*}
{\Sigma}\SET{\kappa(w)\mid w\in E}
\end{gather*}
In practice it is of course useful to break symmetries.  For example
the set $V=\SET{w_1,\ldots,w_n}$ should be ordered using depth in the
tree as primary sort key, minimal element as secondary sort key, and
tag as tertiary sort key.  It is easy to combine these keys together
using an appropriate base, so that elements in $V$ must have strictly
increasing sort keys\ldots\ I think!

\end{document}
