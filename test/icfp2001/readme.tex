\documentclass{article}
\usepackage{amsmath,amstext,amsfonts}
\newcommand{\NAT}{{\mathbb N}}
\newcommand{\SET}[1]{{\{{#1}\}}}
\newcommand{\COST}{{\textsf{cost}}}
\newcommand{\TO}{\rightarrow}
\newcommand{\ROOT}{{\textsf{root}}}
\newcommand{\MOTHER}{{\textsf{mother}}}
\newcommand{\DAUGHTERS}{{\textsf{daughters}}}
\newcommand{\DOWN}{{\textsf{down}}}
\newcommand{\EQDOWN}{{\textsf{eqdown}}}
\newcommand{\UNION}{\cup}
\newcommand{\TUP}[1]{{\langle{#1}\rangle}}
\newcommand{\EPS}{\epsilon}
\newcommand{\AT}[2]{@^{#1}_{#2}}
\newcommand{\CONVEX}{{\textsf{convex}}}
\newcommand{\IMPLIES}{\Rightarrow}
\newcommand{\DEPTH}{{\textsf{depth}}}
\newcommand{\INTER}{\cap}
\makeatletter
\newcommand{\SEQUNION}{{%
  \setbox0\hbox{$\cup$}%
  \@tempdima\wd0\relax
  \hbox{\raise 2.5pt\hbox to0pt{\hbox to\@tempdima{\hss\tiny$\prec$\hss}\hss}\box0}}}
\makeatother
\title{Progamming Contest}
\setlength{\parskip}{\medskipamount}
\begin{document}
\maketitle

The basic idea, due to Andreas, is that the input document gives us an
upper bound on the number of elements $n_E$ that may occur in the
output document.

An input document consists of a finite set $C$ of data items.  These
data items are numbered by successive integers: we capture this by a
function $\pi:C\rightarrow\NAT$ which is a bijection between $C$ and
$\SET{1\ \ldots\ n_C}$ where $n_C=|C|$.  Furthermore, we assume given
a set $A$ of attributes and an assignment $\alpha_C:C\times A
\rightarrow \NAT$.

We assume given a set $T$ of tags and we define the cost of a tag by a
function $\COST_T:T\TO\NAT$:
\begin{center}
\begin{tabular}{c@{\hspace{1cm}}l@{}l}
\verb+<b> ... </b>+ & $\COST_T(\texttt{b})$ & ${}=7$\\
\verb+<pl> ... </pl>+ & $\COST_T(\texttt{pl})$ & ${}=9$
\end{tabular}
\end{center}
an upper bound for the number of element is:
\begin{gather*}
\lfloor{{\Sigma}\SET{\COST_T(e)\mid e\text{ in document}}/7}\rfloor
\end{gather*}
it may be possible to derive a tighter bound when e.g.\ the input
document contains redundant elements etc\ldots

Our formalization of the output document considers the set $C$ of data
items ($|C|=n_C$) and a set $E$ of elements ($|E|=n_E$).  We write
$V=E\uplus C$ for the set of nodes in the tree representation of the
document.  We also assume an additional node $\ROOT$ to serve as the
root node of the document.  We write $V^*=V\uplus\ROOT$.  Not all
elements in $E$ will necessarily be used (hopefully not): in our
formalization, these unused elements will be empty and immediate
daughters of the $\ROOT$.  We consider the following functions:
\begin{align*}
\MOTHER &: V\TO V^*\\
\DAUGHTERS &: V^*\TO 2^V\\
\DOWN &: V^*\TO 2^V\\
\EQDOWN &: V^*\TO 2^{V^*}
\end{align*}
The treeness condition is satisfied when $\forall w\in V^*$ and
$\forall w'\in V$:
\begin{align*}
\EQDOWN(w)&=\SET{w}\uplus\DOWN(w)\\
\DOWN(w)&={\UNION}\SET{\EQDOWN(w')\mid w'\in\DAUGHTERS(w)}\\
w=\MOTHER(w')\ &\equiv\ w'\in\DAUGHTERS(w)\\
V&={\uplus}\SET{\DAUGHTERS(w)\mid w\in V^*}
\end{align*}
Posing $V=\SET{w_1,\ldots,w_n}$,
equation $\DOWN(w)={\UNION}\SET{\EQDOWN(w')\mid w'\in\DAUGHTERS(w)}$
can be implemented by the \emph{selection union} constraint:
\begin{gather*}
\DOWN(w)={\UNION}\TUP{\EQDOWN(w_1),\ldots,\EQDOWN(w_n)}[\DAUGHTERS(w)]
\end{gather*}

We introduce an additional fictional tag $\EPS$ to be assigned to an
element which is not used.  We write $T^*=T\uplus\SET{\EPS}$.  We
introduce the function $\tau:E\TO T^*$ to represent an assignment of
tags to elements.

Each attribute $a\in A$ has a default value $\delta(a)$.  We introduce
the function $\alpha:V^*\times A\TO\NAT$ to represent an assignment of
attribute values to nodes.
\begin{gather*}
\alpha(w,a) = \left\{
\begin{array}{l@{\qquad}l}
\alpha_C(w,a) & \text{for } w\in C\\
\delta(a) & \text{for } w=\ROOT\\
\AT{\tau(w)}{a}(\alpha(\MOTHER(w),a)) & \text{for } w\in E
\end{array}
\right.
\end{gather*}
where $\AT{t}{a}:\NAT\TO\NAT$ is the function which computes the value
of attribute $a$ inside a tag $t$ given its value outside.  For most
attributes it is the identity function.

There are two difficulties in the last equation: (1) computing
$\alpha$ at the unknown point $(\MOTHER(w),a)$ and (2) computing
$\AT{\tau(w)}{a}$ at the unknown point $\alpha(\MOTHER(w),a)$.  This
can be achieved using the following reduction:
\begin{align*}
\alpha(w',a) &= \TUP{\alpha(w_1,a),\ldots,\alpha(w_n,a)}[w']\\
\AT{t}{a}(x) &= \TUP{\AT{t}{a}(1),\ldots,\AT{t}{a}(k_a)}[x]
\end{align*}
where we assume that the possible values for attribute $a$ range from
$1$ to $k_a$.

Furthermore, for all $w''\in C$ it must hold that
\begin{gather*}
\alpha(w'',a) = \alpha(\MOTHER(w''),a)
\end{gather*}

We write $S$ for the data range $\SET{1,\ldots,n_C}$ and introduce the
scope function $\sigma:V^*\TO 2^S$.  For all $w\in V^*$, its scope is
an interval:
\begin{gather*}
\CONVEX(\sigma(w))
\end{gather*}
furthermore, it can be computed inductively:
\begin{gather*}
\sigma(w) = \left\{
\begin{array}{l@{\hspace{1cm}}l}
S & \text{if } w=\ROOT\\
\SET{\pi(w)} & \text{if } w\in C\\
{\UNION}\SET{\sigma(w')\mid w'\in\DAUGHTERS(w)} & \text{if } w\in E
\end{array}
\right.
\end{gather*}
An element $w\in E$ is unused iff it is empty.  Furthermore, in that
case it is an immediate daughter of the $\ROOT$.
\begin{align*}
\tau(w)&=\EPS\ &\equiv\ |\sigma(w)|=0\\
\tau(w)&=\EPS\ &\IMPLIES\ \ROOT=\MOTHER(w)
\end{align*}
Finally, we introduce a cost function $\kappa:E\TO\NAT$:
\begin{gather*}
\kappa(w) = \TUP{\COST_T(t_1),\ldots,\COST_T(t_k)}[\tau(w)]
\end{gather*}
where we pose $T^*=\SET{t_1,\ldots,t_k}$ and $\COST_T(\EPS)=0$.
The goal is to minimize the total cost:
\begin{gather*}
{\Sigma}\SET{\kappa(w)\mid w\in E}
\end{gather*}
In practice it is of course useful to break symmetries.  For example
the set $V=\SET{w_1,\ldots,w_n}$ should be ordered using depth in the
tree as primary sort key, minimal element as secondary sort key, and
tag as tertiary sort key.  It is easy to combine these keys together
using an appropriate base, so that elements in $V$ must have strictly
increasing sort keys\ldots\ I think!

\section*{Breaking Symmetries.}
Let's now consider some methods for breaking symmetries.

\paragraph*{Depth Method.}
Posing $E^*=E\uplus\SET{\ROOT}$, we introduce a function
$\DEPTH:E^*\TO\NAT$ to represent the depth of an element in the tree
representation:
\begin{align*}
\DEPTH(\ROOT) &= 0\\
\DEPTH(w) &= 1+\DEPTH(\MOTHER(w))\qquad\text{for }w\in E
\end{align*}
As usual the latter equation translates into:
\begin{gather*}
\DEPTH(w) = 1+
\TUP{\DEPTH(\ROOT),\DEPTH(w_1),\ldots,\DEPTH(w_{n_E})}[\MOTHER(w)]
\end{gather*}
for the obvious appropriate encoding of $E^*$.  Our first symmetry
break is:
\begin{gather*}
\DEPTH(w_1)\leq\cdots\leq\DEPTH(w_{n_E})
\end{gather*}
in other words, elements are ordered by increasing depth.  Thus $w_1$
will always be a child of $\ROOT$, and earlier elements are used for
higher levels in the tree and later elements for deeper levels.

\paragraph*{Sequenced Union Method.}
A more clever method, is to say that the mother of an element must
always precede it in the order $\ROOT \prec w_1 \prec \cdots \prec
w_{n_E}$:
\begin{gather*}
\MOTHER(w)\prec w\qquad\text{for } w\in V
\end{gather*}
and additionally if $w$ is mother of $w'$ and $w''$ and
$w'\prec w''$ then all the elements in $\DOWN(w')$ must precede all the
elements in $\DOWN(w'')$.  For this, we introduce the functions
$\DOWN_E:E\TO 2^E$, $\EQDOWN_E:E\TO 2^E$, and $\DAUGHTERS_E:E\TO 2^E$
defined by:
\begin{align*}
\DOWN_E(w) &= E\INTER\DOWN(w)\\
\EQDOWN_E(w) &= E\INTER\EQDOWN(w)\\
\DAUGHTERS_E(w) &= E\INTER\DAUGHTERS(w)
\end{align*}
and the following constraint must be satisfied:
\begin{gather*}
\DOWN_E(w) =
{\SEQUNION}\TUP{\EQDOWN_E(w_1),\ldots,\EQDOWN_E(w_{n_E})}[\DAUGHTERS_E(w)]
\end{gather*}
This is an application of the \emph{sequenced union selection}
constraint.  It combines the \emph{union selection} constraint with
the sequentiality constraint.

Note that scope should also be constrained using the sequenced union
selection constraint:
\begin{gather*}
\sigma(w) = {\SEQUNION}\TUP{\sigma(w_1),\ldots,\sigma(w_n)}[\DAUGHTERS(w)]
\end{gather*}
i.e.\ not only is the scope of the mother formed from the union of the
scopes of its daughters, but these scopes must also occur in sequence
according to the precedence order $\prec$ of elements (i.e.\ the
indices used to encode them).

\paragraph*{Stacking Method.}
This last method is concerned with symmetries of the following kind:
\begin{center}
\verb+<B><I>...</I></B>+\linebreak
\verb+<I><B>...</B></I>+
\end{center}
whenever $w$ is mother of a single child $w'$ then $w$ must be
assigned a \emph{smaller} tag than $w'$.  Thus, here we must assume a
total order $\prec$ on tags, and the constraint is:
\begin{gather*}
w=\MOTHER(w')\ \wedge\ |\DAUGHTERS(w)|=1\quad
\IMPLIES\quad \tau(w)\preceq\tau(w')
\end{gather*}
For this to work, it must be the case that when switching the 2 tags
is semantically significant then the configuration which does not
satisfy the above constraint is redundant.  I don't know if this is
the case.  Note that we used $\preceq$ rather than the stricter
$\prec$ to allow for such stackings as \verb+<U><U>...</U></U>+.
For this method to work, $\ROOT$ must be assigned some fictitious and lowest
tag.

\end{document}
