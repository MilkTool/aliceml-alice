\documentclass{article}
\usepackage{german}
% \usepackage[latin1]{inputenc}

\setlength{\textwidth}{14cm}
\setlength{\oddsidemargin}{1cm}

\begin{document}
\begin{center}
\begin{huge}Benutzungshinweise f"ur die Alice-GTK-Bindings\end{huge}

Entwurf vom 16.10.02

Robert Grabowski

\end{center}



\section{Einleitung}
Mit dieser Schnittstelle ist es m"oglich, die GTK+-2-Bibliothek sowie das
GnomeCanvas unter Alice zu nutzen. Dieses Dokument beschreibt, wie die
Schnittstelle aufgebaut ist und welche Besonderheiten unter Alice zu beachten
sind. Die eigentliche Dokumentation der GTK-Bibliothek selbst findet man unter
\texttt{http://www.gtk.org/}; die Dokumentation des GnomeCanvas-Widgets unter
\texttt{http://developer.gnome.org/doc/API/libgnomeui/}.



\section{Einbinden der Schnittstelle}
Die Schnittstelle besteht aus den Dateien ???. Die Komponente ``Gtk'' stellt
die Funktionalit"at der GTK-Bibliothek zur Verf"ugung, die Komponente ``Gdk''
die der GDK-Bibliothek und die Komponente ``GnomeCanvas'' die des 
GnomeCanvas-Widgets. In Alice-Programme k"onnen diese mit der Deklaration
\texttt{import structure Gtk from ``Gtk''} eingebunden werden (analog f"ur
die anderen Komponenten).



\section{Enum-Werte}

Alle Werte aus Enumerations-Typen (C-Aufz"ahlungs-Typen) sind in der
Schnittstelle als \texttt{datatype}s mit nullstelligen Konstruktoren
deklariert.

Dabei tragen die Konstruktoren die selben Namen wie die entsprechenden
Enum-Konstanten unter C, jedoch ohne das Pr"afix \texttt{GTK\_},
\texttt{GDK\_} bzw. \texttt{GNOME\_CANVAS\_}. 

Beispiele:

Der Aufz"ahlungstyp GtkWindowType ist im Binding deklariert als

\texttt{datatype GtkWindowType = WINDOW\_TOPLEVEL | WINDOW\_POPUP}

Das Pr"afix bei den Konstruktornamen wird allerdings \textit{nicht} entfernt,
falls der Name ohne das Pr"afix mit einer Ziffer beginnen und damit
einen ung"ultigen Bezeichnernamen besitzen w"urde.

Beispiel:

\texttt{GDK\_2BUTTON\_PRESS} hei"st unter Alice 
  \texttt{Gdk.GDK\_2BUTTON\_PRESS}

Manchmal kommt es vor, dass mehrere Konstanten in einem C-Enum-Typ denselben
Wert haben. Daher sind auch im Binding beispielsweise die Konstruktoren
\texttt{ANCHOR\_SOUTH\_WEST} und \texttt{ANCHOR\_SW} gleichbedeutend
und k"onnen beide als Funktionsparameter verwendet werden. Wenn eine Funktion
jedoch einen Enum-Wert zur"uckliefert, wird sie immer nur denjenigen
Konstruktor verwenden, dessen Name in der C-Enum-Deklaration weiter oben
auftaucht (siehe GTK-Manual); in diesem Fall also \texttt{ANCHOR\_SOUTH\_WEST}.



\section{Funktionen}
Grunds"atzlich sind alle GTK-, GDK- und GnomeCanvas-Funktionen nutzbar, d.h.
alle Funktionen, die mit \texttt{gtk\_}, \texttt{gdk\_} oder
\texttt{gnome\_canvas\_} beginnen.

\subsection{Funktionsname}
Die Alice-Funktionsnamen entsprechen nicht den C-Funktionsnamen, sondern sind
nach einer bestimmten Konvention umgewandelt:
\begin{enumerate}
\item Das Pr"afix (\texttt{gtk\_}, \texttt{gdk\_} bzw. 
      \texttt{gnome\_canvas\_}) wird entfernt,
\item jeder Unterstrich (\_) wird entfernt; der jeweils darauffolgende
      Buchstabe wird zu einem Gro"sbuchstaben.
\end{enumerate}

Beispiel: Die Funktion \texttt{gtk\_window\_set\_title} hei"st unter Alice
\texttt{Gtk.windowSetTitle}.

\subsection{Funktionsargumente}
Die C-Typen der Argumente und des R"uckgabewerts werden entsprechend folgender
Tabelle auf Alice-Typen abgebildet:

\begin{tabular}{|c|c|}
\hline \textbf{C-Typ}     & \textbf{Alice-Typ} \\
\hline void               & unit       \\
\hline (ganzzahliger Typ) & int        \\
\hline (Flie"skomma-Typ)  & real       \\
\hline gboolean           & bool       \\
\hline char*/gchar*       & string     \\
\hline (enum-Typ)         & (datatype) \\
\hline t[]                & t array    \\
\hline t*                 & object     \\
\hline GList*/GSList*     & object list\\
\hline .../valist         & arg list\\
\hline
\end{tabular}

Anmerkungen:

\begin{itemize}
\item F"ur jeden C-Enum-Typ gibt es eine entsprechenden datatype-Deklaration
      unter Alice. Mehr dazu in Abschnitt ???.

\item Der Typ \texttt{object} ist ein abstrakter Datentyp, der lediglich einen
      Zeiger auf beliebige Datenstrukturen darstellt. Werte vom Typ
      \texttt{object} k"onnen auf Gleichheit "uberpr"uft werden. Der Typ ist
      in allen Komponenten vorhanden und gleich.

\item Es gibt Funktionen, die in C eine variable Anzahl von Parametern nehmen.
      Diesen Funktionen wird in Alice eine \texttt{arg list} "ubergeben.
      Mehr dazu in Abschnitt ???.

\end{itemize}

Hat die Funktion mehrere Argumente, so werden diese als Tupel "ubergeben.

Beispiel: Die C-Funktion
 
\texttt{void gtk\_window\_set\_title(GtkWindow*, gchar*)}

hat unter Alice die Signatur

\texttt{val windowSetTitle : object * string -> unit}

\subsection{Ein- und Ausgabe-Argumente}

Bei den meisten Bibliotheksfunktionen werden alle Argumente \textit{by value}
"ubergeben. Die Funktionsausgabe wird "uber den R"uckgabewert "ubergeben.
Zu beachten ist, dass auch Argumente von einem Typ wie \texttt{GtkWindow*}
Werteparameter sind, da die Funktion evtl. die Informationen in der
Datenstruktur \texttt{GtkWindow} "andert, nicht jedoch den Zeiger selbst.

Manche Funktionen erwarten jedoch keine Werte, sondern Zeiger auf
Speicherstellen, an die sie ihre Ausgaben schreiben k"onnen. Ein solcher
Ausgabeparameter ist immer ein Zeiger auf ein Wert eines der folgenden Typen:

\begin{itemize}
\item Zahl (z.B. gfloat*, gint*)
\item Enumtyp (z.B. ???)
\item String (z.B. gchar**)
\item Liste (z.B. GList**, GSList**) (???)
\item Zeiger auf Datenstruktur (z.B. GtkWindow**)
\end{itemize}

Da Alice dieses Konzept nicht unterst"utzt\footnote{Eigentlich schon, mit
Referenzen, aber das w"urde die Schnittstelle wesentlich unhandlicher machen.},
werden diese Ausgabeparameter zusammen mit dem eigentlichen R"uckgabewert
als Tupel zur"uckgegeben. Ist der eigentliche R"uckgabewert vom Typ unit,
entf"allt er aus dem Tupel.

Beispiel:

\texttt{void gtk\_image\_get\_stock(GtkImage*, gchar**, GtkIconSize*)}

wird zu 

\texttt{val imageGetStock : object -> string * GtkIconSize}

In seltenen F"allen kommt es vor, dass Ausgabeparameter auch als Eingabe-
Parameter dienen. Daher gibt es f"ur jede Funktion, die mindestens
einen Ausgabeparameter hat, noch eine zweite Variante, in der es f"ur
alle Ausgabeparameter zus"atzlich noch Eingabeparameter gibt.

Beispiel: Obige Funktion imageGetStock hat noch eine zweite Variante, und zwar

\texttt{val imageGetStock' : object * string * GtkIconSize 
        -> string * GtkIconSize}


\subsection{Funktionen mit variablen Argumenten}

???



\section{Konzepte}

\subsection{Die Hauptschleife}

Sobald die Gtk-Struktur geladen wurde, wird automatisch gtk\_init aufgerufen.
Au"serdem wird automatisch ein Thread gestartet, in dem die GTK-
Signalverarbeitungs-Schleife l"auft.

Dadurch werden Widgets angezeigt, sobald z.B. \texttt{Gtk.widgetShow}
aufgerufen wurde. Es ist also nicht notwendig (und auch nicht m"oglich), 
manuell im Alice-Programm gtk\_init oder gtk\_main aufzurufen.

\subsection{Event-Handling}

Um eine Callback-Funktion an ein Widget und ein Signal zu binden, steht die
Funktion \texttt{Gtk.signalConnect} zur Verf"ugung. Die Signatur sieht so aus:

\texttt
{val signalConnect : object * string * (object * arg list -> unit) -> int}

Der erste Parameter ist das Widget, an dass die Callback-Funktion gebunden
werden soll, der zweite Parameter ist der Name des Ereignisses, und der
dritte Parameter ist schlie"slich die Callback-Funktion selbst.
Der R"uckgabewerte ist die ``connection id''. Diese wird nur ben"otigt, wenn
die Funktion sp"ater wieder getrennt werden soll.

Die Callback-Funktion wiederum erh"alt das Widget, welches das Ereignis
ausgel"ost hat, sowie eine Liste von Argumenten. (???) 

Beispiel:

\texttt{val button = Gtk.buttonNewWithTitle(``Click me'')\\
        fun showMessage \_ = print ``button clicked!''\\
        val connid = Gtk.signalConnect(button, ``clicked'', showMessage)}


\subsection{Typ-"Uberpr"ufungen f"ur \texttt{object}s}

GTK+ besitzt eine eigene Klassenhierarchie, wobei \texttt{GObject} die
Basisklasse f"ur alle GTK+-Objekte ist. Ferner besitzt die Bibliothek ein
eigenes Typsystem, mit dem z.B. die Klasse eines Objekts zur Laufzeit 
"uberpr"uft werden kann. Genau dies tut die Bibliothek, so dass bei einem
Typenkonflikt eine Fehlermeldung direkt auf dem Bildschirm ausgegeben wird.

Wenn die Native-DLLs der Schnittstelle mit der Compiler-Option \texttt{-DDEBUG}
"ubersetzt werden, wird vor jedem Bibliotheks-Funktionsaufruf eine eigene
Typ"uberpr"ufung f"ur Werte vom Typ \texttt{object} vorgenommen.

Falls die Funktion ein Objekt einer bestimmten Klasse A
verlangt, falls das "ubergebene \texttt{object} ein Objekt der Klasse B ist
und falls B weder A noch eine abgeleitete Klasse von A ist, wird eine Ausnahme
vom Typ Gtk.TypeMismatch geworfen. Diese ist deklariert als
\texttt{exception TypeMismatch of string}; der string enth"alt eine
Fehlermeldung.

Falls das DEBUG-Flag nicht gesetzt ist, bleibt die Typ"uberpr"ufung aus
Performance-Gr"unden aber abgeschaltet.



\section{Besonderheiten und Ausnahmen}

\subsection{Konstanten}

Die Struktur ``Gtk'' stellt folgende Konstanten zu Verf"ugung:
\begin{itemize}
\item \texttt{val NULL : object}\\
      stellt einen NULL-Zeiger zur Verf"ugung. Dies wird f"ur einige Funktionen
      ben"otigt.
\item \texttt{val TRUE : int}\\
      \texttt{val FALSE : int}\\
      entsprechen den gleichnamigen Konstanten aus der C-Bibliothek.
      Achtung: Das Alice-Pendant zu \texttt{gboolean} ist \texttt{bool}, man
      kann also fast immer das gew"ohnliche ``true'' und ``false'' benutzen.
      Die Konstanten Gtk.TRUE und Gtk.FALSE sind f"ur Funktionen gedacht, die
      ein \texttt{int} verlangen, aber trotzdem ein TRUE oder FALSE erwarten.
\end{itemize}

\subsection{String-Listen und -Arrays}

Bei manchen Funktionen wird ein Parameter vom Typ \texttt{GList*} oder 
\texttt{GSList*} nicht auf den Typ \texttt{object list}, sondern auf den Typ
\texttt{string list} abgebildet.
So erwartet zum Beispiel Gtk.comboSetPopdownStrings eine \texttt{string list}.

\subsection{Konstruktor-Funktionen}

Einige Datenstrukturen wie GdkColor werden nicht "uber eine 
Konstruktor-Funktion erstellt, sondern werden unter C einfach direkt als
Variable deklariert, was in Alice nicht m"oglich ist. F"ur andere
Datenstrukturen wiederum gibt es Konstruktoren, die leider sehr unpraktisch
f"ur den Gebrauch in Alice sind. Deshalb bietet die Schnittstelle folgende
zus"atzliche/ge"anderte Konstruktoren an:

\begin{itemize}
\item val Gdk.colorNew : int * int * int -> object\\
      liefert einen Zeiger auf eine neu erstellte GdkColor-Struktur, die mit
      den angebenen Werten f"ur rot, gr"un und blau initialisiert ist.
\item val Gdk.rectangleNew : int * int * int * int -> object\\
      liefert einen Zeiger auf eine neu erstellte GdkRectangle-Struktur, die
      mit den angegeben Werten f"ur x1, y1, x2 und y2 initialisiert ist.
\item val Gtk.textIterNew : unit -> object\\
      liefert einen Zeiger auf einen neuen, unintialisierten GtkTextIter.
\item val Gtk.treeIterNew : unit -> object\\
      liefert einen Zeiger auf einen neuen, unintialisierten GtkTreeIter.
\item val Gtk.treeStoreNew : unit -> object\\
      liefert einen Zeiger auf einen GtkTreeStore. Die Baumrepr"asentation
      enth"alt dabei nur eine Spalte von Typ string.
\item val GnomeCanvas.itemNew : object * int -> object\\
      verh"alt sich wie die urspr"ungliche Version mit der Ausnahme, dass
      keine Attribute angegeben werden k"onnen. Dies kann mit
      GnomeCanvas.itemSetValist geschehen.
\end{itemize}

\subsection{Zus"atzliche Funktionen}

Folgende Funktionen sind zus"atzlich verf"ugbar:

\begin{itemize}
\item val Gtk.treeViewGetSelectedString : object -> string\\
      liefert den Eintrag, der im angegeben Baum markiert ist, oder ````, falls
      nichts markiert ist.
\item val GnomeCanvas.pointsSetCoords : object * int * int\\
      Beispiel: pointsSetCoords(p, 4, 100) setzt die 4. Koordinate im
      GnomeCanvasPoints-Array p auf 100 (Z"ahlung beginnt bei 0).
\end{itemize}

\subsection{Nicht verf"ugbare Funktionen}

Folgende Funktionen sind nicht verf"ugbar:
\begin{itemize}
\item Alle Funktionen, die nur als C-Makro definiert sind.
\item gtk\_true und gtk\_false. Ihr Verhalten l"asst sich aber mit
      \texttt{(fn () => Gtk.TRUE)} bzw. \texttt{(fn () => Gtk.FALSE)}
      nachbilden.
\item gtk\_init, gtk\_init\_check und gtk\_main (siehe ???).
\end{itemize}

\end{document}             

