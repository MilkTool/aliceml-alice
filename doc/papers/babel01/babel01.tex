%%%
%%% Author:
%%%   Leif Kornstaedt <kornstae@ps.uni-sb.de>
%%%
%%% Copyright:
%%%   Leif Kornstaedt, 2001
%%%
%%% Last change:
%%%   $Date$ by $Author$
%%%   $Revision$
%%%

% Upper-case    A B C D E F G H I J K L M N O P Q R S T U V W X Y Z
% Lower-case    a b c d e f g h i j k l m n o p q r s t u v w x y z
% Digits        0 1 2 3 4 5 6 7 8 9
% Exclamation   !           Double quote "          Hash (number) #
% Dollar        $           Percent      %          Ampersand     &
% Acute accent  '           Left paren   (          Right paren   )
% Asterisk      *           Plus         +          Comma         ,
% Minus         -           Point        .          Solidus       /
% Colon         :           Semicolon    ;          Less than     <
% Equals        =           Greater than >          Question mark ?
% At            @           Left bracket [          Backslash     \
% Right bracket ]           Circumflex   ^          Underscore    _
% Grave accent  `           Left brace   {          Vertical bar  |
% Right brace   }           Tilde        ~

\documentclass{entcs}
\usepackage{entcsmacro}

\newcommand{\kw}[1]{{\bfseries#1}}

\def\arraystretch{1}

\begin{document}

\def\lastname{Kornstaedt}
\begin{frontmatter}
  \title{Alice in the Land of Oz}
  \subtitle{An Interoperability-based Implementation of a \\
    Functional Language on Top of a Relational Language}
  \author{Leif Kornstaedt}
  \address{\href{mailto:kornstae@ps.uni-sb.de}{%
	     \texttt{\normalshape kornstae@ps.uni-sb.de}} \\
	   Programming Systems Lab \\
	   Universit\"{a}t des Saarlandes \\
	   Saarbr\"{u}cken, Germany}
  \begin{abstract}
    This paper reports practical experience in implementing Alice, an
    extension of Standard ML, on top of an existing implementation of Oz.
    This approach yields a high-quality implementation with little effort.
    The combination is an advanced programming system for both Oz and
    Alice, which offers more than either language on its own.
  \end{abstract}
\end{frontmatter}


%=======================================================================
\section{Introduction}
%=======================================================================

The language Oz and its implementation Mozart~\cite{Mozart} constitute an
advanced programming system, targeted at concurrent constraint programming and
open distributed computing.  It is used in areas as diverse as combinatorial
problem solving and scheduling, natural language processing (parsing,
semantics representation, and inference), multi-agent systems, and
collaborative tools.

%--** andersherum: erst die key features, dann die Schwachpunkte
However, Oz has the reputation of being an exotic language, preventing
it from being more widely used.  In part, it owes this reputation to its
syntax.  To make the achievements of Oz available to a larger community,
we are integrating its key features into a wide-spread functional language:
In a first step, we have extended Standard ML by data-flow-driven concurrency
and components with lazy dynamic linking.  We call this language \emph{Alice}.

This paper describes an implementation of Alice on top of Oz.  This approach
maximizes reuse of the existing implementation technology developed for Oz.
By making Alice-Oz-interoperability a major goal, we immediately obtain the
following benefits:

\begin{itemize}
\item It becomes straightforward to implement Alice's primitives in Oz.
\item Since Alice can import components implemented in Oz, all features
  of Oz' runtime system are immediately available to Alice, such as
  constraints~\cite{ConstraintPropagation}, computation spaces for
  speculative computation~\cite{ConstraintServices}, persistence, and
  distribution~\cite{Distribution}.
\item Since Oz can import components implemented in Alice, users can
  evaluate Alice while keeping their existing Oz codebase, implementing
  some component or other in Alice, before maybe migrating entire projects.
\item Since Alice is backward-compatible with SML, existing SML libraries
  can be made directly accessible to Oz programmers, by compiling them with
  the Alice compiler.
\end{itemize}

\noindent In short, the resulting system offers benefits to both the
existing Oz community and to existing SML programmers.

%--** contributions/case study + lessons learned: positionieren, siehe CfP

This paper is structured as follows: Section~\ref{cast} presents the
languages Oz and Alice.  Section~\ref{terminology} introduces some
terminology and motivates our approach to interoperability, before
Section~\ref{mapping} gives a detailed account of how we put this into
practice.  The implementation is described in Section~\ref{implementation}.
We discuss the pros and cons of our approach in Section~\ref{discussion}
and conclude the paper with an outlook on future work in Section~\ref{outlook}.


%=======================================================================
\section{The Cast: Oz and Alice}\label{cast}
%=======================================================================

This section introduces the protagonists of our interoperability story:
the language Oz, the programming system Mozart which implements Oz, and
the language Alice.

%-----------------------------------------------------------------------
\subsection{The Language Oz}
%-----------------------------------------------------------------------

The language Oz is defined in terms of a small sub-language, Core Oz,
and a number of syntactic extensions which are defined by translation
to Core Oz~\cite{OzNotation}.  Like Oz, Core Oz is a dynamically typed,
concurrent, non-backtracking relational language.  Its syntax is summarized
in Fig.~\ref{coreozsyntax}, using the symbol~$i$ for integers, $a$~for atoms,
and $x$, $y$, and~$z$ for variables.\footnote{We omit the constraint
extensions for the purpose of this paper.}

\begin{figure}
  \caption{Syntax of Core Oz.}\label{coreozsyntax}
  \centerline{
    \begin{tabular}{rcll} \hline
      \multicolumn{4}{l}{\bfseries Statements} \\
      $s$ & ::= & $s_1$ $s_2$
		& {\small sequential composition} \\
	  & $|$ & \kw{local} $x$ \kw{in} $s$ \kw{end}
		& {\small local declaration} \\
	  & $|$ & $x$ = !!$y$
		& {\small future extraction} \\
	  & $|$ & $x$ = $y$
		& {\small unification} \\
	  & $|$ & $x$ = $t$
		& {\small tell value} \\
	  & $|$ & $x$ = $y$.$z$
		& {\small record selection} \\
	  & $|$ & \{NewName $x$\}
		& {\small name creation} \\
	  & $|$ & \kw{case} $x$ \kw{of} $t$ \kw{then} $s_1$
		  \kw{else} $s_2$ \kw{end}
		& {\small conditional} \\
	  & $|$ & \kw{proc} \{$x$ $y_1 \ldots y_n$\} $s$ \kw{end}
		& {\small abstraction ($n \ge 0$)} \\
	  & $|$ & \{$x$ $y_1 \ldots y_n$\}
		& {\small application ($n \ge 0$)} \\
	  & $|$ & \kw{thread} $s$ \kw{end}
		& {\small thread creation} \\
	  & $|$ & \{NewCell $x$ $y$\}
		& {\small cell creation} \\
	  & $|$ & \{Exchange $x$ $y$ $z$\}
		& {\small cell exchange} \\
	  & $|$ & \{Raise $x$\}
		& {\small exception raising} \\
	  & $|$ & \kw{try} $s_1$ \kw{catch} $x$ \kw{then} $s_2$ \kw{end}
		& {\small exception handling} \\
      \multicolumn{4}{l}{\bfseries Terms} \\
      $t$ & ::= & $i$
		& {\small integers} \\
	  & $|$ & $l$($f_1: x_1 \ldots f_n: x_n$)
		& {\small records} \\
      $l$ & ::= & $x$ $|$ $a$ $|$ \kw{true} $|$ \kw{false} $|$ \kw{unit}
		& {\small literals} \\
      $f$ & ::= & $l$ $|$ $i$
		& {\small features} \\
      \hline
    \end{tabular}
  }
\end{figure}

Core Oz supports the following data types:
\begin{description}
\item[Transients] are placeholders for unknown values.  Transients provide
  for data-flow synchronization:  Threads block when they need the values
  of transients for computations.  Binding a transient replaces the transient
  by its value, resuming all threads suspending on it.

  One kind of transient is the \emph{logic variable}, similar to Prolog's.
  A read-only view of a logic variable is a \emph{future}.  A \emph{by-need}
  future carries a nullary procedure which is applied when the future's
  value is required, with the effect of replacing the future by a new
  value computed by the procedure.  By-need futures are used to model lazy
  computations.\footnote{By-need futures are created by a primitive operation
  not depicted in the syntax.}
\item[Atoms] are internalized strings, like Lisp symbols.
\item[Names] represent structureless identities.  An infinite number of new
  names can be generated.  The boolean values \kw{true} and \kw{false} are
  predefined names, as is the value \kw{unit} used for synchronization.
  Atoms and names are collectively referred to as \emph{literals}.
\item[Records] are structured values consisting of a \emph{label}~$l$ and
  any number of \emph{subtrees}~$x_i$, accessible through the unique
  \emph{features}~$f_i$ of the record.  The label~$l$ serves to model
  algebraic data types.  If $y$ is a record of the form $l$($f_1: x_1
  \ldots f_n: x_n$) and $z$ = $f_i$, then the record selection statement
  $x$ = $y$.$z$ unifies $x$ with $x_i$.  Records with integer features
  starting from~$1$ are called \emph{tuples}.  Lists are defined as either
  the atom `nil' or a binary tuple with label~`$|$', whose second element
  is a list.
\item[Procedures] take any number of arguments.  All arguments are input
  arguments; output arguments are modelled by supplying a logic variable as
  input argument and binding it in the procedure's body.  Procedures are
  first-class data structures.
%--** lexically scoped
\item[Cells] are the only stateful entities in Core Oz.  A cell holds a
  reference to an Oz value and supports an atomic exchange operation.
%--** Constraints
\end{description}

\noindent While Core Oz is a relational language, meaning that `='
is unification, not assignment, full Oz supports the functional and
object-oriented programming paradigms with dedicated syntax.  This syntax
is translated to Core Oz:  Functions, for instance, are procedures with an
additional argument for output (unification allows for bidirectional data
flow).  Due to the translation, functions can be called as procedures and
vice-versa.

Oz uses records as modules, without any dedicated syntax:  Modules are just
a convention for structuring applications.  Oz applications are deployed
as a number of components.  A component is defined using the \kw{functor}%
\footnote{Oz calls components \emph{functors}; to avoid confusion with
SML functors, we will only be using the term \emph{component} in this paper.}
construct (see Fig.~\ref{ozcomponents}(a) for an example).  A component
consists of an import declaration, specifying the components to link to as
well as their locations, in the form of URLs; an export declaration, naming the
values to export (as a record); and a sequence of declarations and statements
to execute in order to build the export module.  Components are translated to
Core Oz data structures; the exact translation is beyond the scope of this
paper.

\begin{figure}
  \caption{(a)~A sample Oz component `Test'.
    (b)~A sample Oz application using `Test'.}
  \label{ozcomponents}
  \centerline{
    \begin{tabular}{@{}ll@{}} \hline
      \multicolumn{1}{c}{(a)} &
      \multicolumn{1}{c}{(b)} \\
      \begin{tabular}{l}
	\kw{functor} \\
	\kw{export} fak: Fak \\
	\kw{define} \\
	\quad \kw{fun} \{Fak N\} \\
	\qquad \kw{if} N == 0 \kw{then} 1 \\
	\qquad \kw{else} N * \{Fak N - 1\} \\
	\qquad \kw{end} \\
	\quad \kw{end} \\
	\kw{end}
      \end{tabular} &
      \begin{tabular}{l}
	\kw{functor} \\
	\kw{import} \\
	\quad Test \kw{at} 'Test' \\
	\quad System \kw{at} 'x-oz://system/System' \\
	\quad Application \kw{at} 'x-oz://system/Application' \\
	\kw{define} \\
	\quad \{System.show \{Test.fak 7\}\} \\
	\quad \{Application.exit 0\} \\
	\kw{end}
      \end{tabular}
    \\ \hline
    \end{tabular}
  }
\end{figure}

%-----------------------------------------------------------------------
\subsection{The Programming System Mozart}
%-----------------------------------------------------------------------

The Mozart System is an implementation of the language Oz.  It is based on
a virtual machine, which in its original conception owes much to the Warren
Abstract Machine~\cite{WAM} designed for Prolog.  The Mozart Virtual Machine
has specifically been designed for Oz~\cite{Amoz,Mehl}, and as such provides
most of the features required by Oz directly:  The garbage-collected memory
store is able to represent all Oz data types directly, in particular including
logic variables, futures, and constraints.  Data-flow synchronization is
implicitly taken care of by the instructions.  The runtime knows how to make
data structures and code persistent, and how to marshal them to other sites
in a distributed environment.

The runtime is started up with the URL of the application component to execute
on the command line.  The machine first loads an initialization component,
which may only import built-in components.  The initialization component
instantiates a \emph{component manager}, which is responsible for lazy
loading and linking of components.  When a component from a specific URL is
first imported, a by-need future is created that will eventually cause the
loading and execution of the corresponding component.  Executing a component
causes it to compute and return a module.  The mapping from URLs to by-need
futures resp.\ modules is maintained in a table internal to the component
manager.  Next, the initialization component dynamically imports and requests
the application component.  The actual application consists of the
side-effects caused by executing the application component (see Fig.~%
\ref{ozcomponents}(b) for an example).

Mozart components are platform-independent:  Currently, Mozart is available
for a wide range of platforms, including Windows, Linux, and MacOS~X.

Oz' component system and component managers are described in~\cite{OzModules}.

%-----------------------------------------------------------------------
\subsection{The Languages Standard ML and Alice}
%-----------------------------------------------------------------------

Standard ML (\emph{SML}) is a strict functional programming language.
As such it is based on the notion of evaluating expressions instead
of executing sequences of statements, uses call-by-value for function
applications, and supports functions as first-class values.  SML is
specifically designed for safety:  Data structures are classified into
immutable and stateful data structures.  Error handling is facilitated
by an exception handling mechanism.  Run-time type errors are prevented
by its static polymorphic type system.

SML is factored into the core language and the module language.  The core
language supports type inference, allowing concise programs with minimal
notational overhead for type annotations.  SML has an advanced module system
with type abstraction and parameterized modules.  Both the static and
dynamic semantics of SML are formally defined~\cite{SMLDefinition}.

One peculiarity of SML is that all constructors and functions are unary,
as opposed to taking an arbitrary number of arguments.  The common idiom
is to model multiple arguments by tupling, although a concise syntax for
defining curried functions is available.

Alice \cite{Alice} is an extension of Standard ML targeted specifically
at open programming.  We chose SML as the base language for Alice for the
following reasons:
\begin{itemize}
\item Due to its clean design, it is relatively easy to devise extensions
  that do not break the soundness of the language.  Its formal definition
  will eventually allow us to model the extensions formally and prove
  their soundness.
\item Standard ML is well-known and has a large user community.  Programmers
  fluent in SML will be able to learn Alice quickly.
\item A lot of teaching material is available for SML.  This material can
  serve as a starting point to teach resp.\ learn Alice.
\end{itemize}

\noindent To compensate for the areas in which we find SML lacking,
we integrated extensions inspired by Oz:
\begin{itemize}
\item Concurrency comes naturally to Open Programming (modelling servers;
  responding to input in graphical user interfaces).  Alice extends SML by
  data-flow driven concurrency.  Logic variables (which are better termed
  \emph{holes} in the context of functional languages) are powerful but
  error-prone:  For increased safety, Alice only provides futures and
  \emph{promises}, which are non-transparent logic variables.
\item Dynamically linked components are mandatory for a truly open
  programming system.  SML's language definition completely omits even
  the practical issues of how programs are to be organized into source
  files and how to support separate compilation.  The existing SML
  implementations address these issues each in their own way, for instance
  Standard ML of New Jersey~\cite{SMLofNJ} has its \emph{Compilation Manager}~%
  \cite{CM}.  None of these approaches provide dynamic linking in a
  satisfactory way.  Alice extends Standard ML by platform-independent
  components \`{a} la Oz.  Fig.~\ref{alicecomponents} shows Alice versions
  of the Oz components depicted in Fig.~\ref{ozcomponents}.  When the compiler
  encounters an import announcement, it loads the referenced component to
  obtain its export signature.  At run-time, the expected signature is checked
  against the actual export signature upon dynamic linking, raising an
  exception upon mismatch.
\end{itemize}

\begin{figure}
  \caption{(a)~A sample Alice component `Test'.
    (b)~A sample Alice application using `Test'.}
  \label{alicecomponents}
  \centerline{
    \begin{tabular}{@{}ll@{}} \hline
      \multicolumn{1}{c}{(a)} &
      \multicolumn{1}{c}{(b)} \\
      \begin{tabular}{l}
	\kw{structure} Test = \kw{struct} \\
	\quad \kw{fun} fak 0 = 1 \\
	\qquad \ $|$ fak n = n * fak (n - 1) \\
	\kw{end}
      \end{tabular} &
      \begin{tabular}{l}
	\kw{import} \kw{structure} Test \kw{from} \char`\"Test\char`\" \\
	\kw{import} \kw{structure} TextIO \\
	\quad \kw{from} \char`\"x-alice:/lib/system/TextIO\char`\" \\
	\kw{import} \kw{structure} OS \\
	\quad \kw{from} \char`\"x-alice:/lib/system/OS\char`\" \\
	\kw{val} \_ = TextIO.print (Test.fak 7) \\
	\kw{val} \_ = OS.Process.exit 0
      \end{tabular}
    \\ \hline
    \end{tabular}
  }
\end{figure}


%=======================================================================
\section{Interoperability: Terminology and Approach}\label{terminology}
%=======================================================================

The design of an interoperability interface has to take two levels into
account: the language level and the implementation level.  At the language
level, we can distinguish between transparent interoperability (programmers
do not see whether the procedures or data they use are foreign) and
non-transparent interoperability (programmers will need to use different
operations when dealing with native vs.\ foreign procedures and data).
At the implementation level, we find direct interoperability (both
languages operate on the same representations of procedures and data) and
marshaling-based interoperability (each language uses its own representations
for procedures and data, which have to be converted at the interoperation
boundaries).

In practice we find all possible combinations:
\begin{description}
\item[Transparent\slash direct] is how C\#~\cite{CSharp} and Visual Basic~7
  interoperate in Microsoft's .NET Common Language Runtime~\cite{DotNet}.
\item[Transparent\slash marshaling-based] is used by the .NET Common Language
  Runtime to access platform-specific dynamically linked libraries (P/Invoke).
\item[Non-transparent\slash direct] interoperation takes place when using
  Java objects from MLj~\cite{MLj}.
\item[Non-transparent\slash marshaling-based] is found in Mozart's
  C/C++ foreign function interface.
\end{description}

\noindent In the following, we will consider the case where one programming
language (termed the \emph{guest language}) is to be implemented on top of
an existing implementation of another language (the \emph{host language}),
sharing the host's runtime system, with bidirectional interoperability.
Let's consider the advantages and drawbacks of the approaches outlined above.
\begin{description}
\item[Transparent vs.\ non-transparent.]  When interoperating
  non-trans\-pa\-rent\-ly, special language extensions or libraries are
  required for dealing with foreign procedures and data.  Transparent
  interoperability makes the strong presupposition of a (bijective) mapping
  which maps the natural way to express something in the guest language to
  the natural way to express that something in the host language.
\item[Marshaling vs.\ direct.]  Marshaling-based interoperability allows
  each language to use the most efficient representation for its data and
  procedures, but causes performance loss at the interoperation boundaries
  due to conversion and copying of data and adaption of calling conventions.
  Direct interoperability does not impose this overhead, but again is much
  harder to design:  When there is no equivalent for some data structure,
  then it has to be \emph{modelled} using the data structures of the host
  language, that is, expressed indirectly.
\end{description}

%-----------------------------------------------------------------------
\subsection{Our Approach}
%-----------------------------------------------------------------------

In our case, the host language is Oz with its implementation Mozart, and the
guest language is Alice.  The following considerations determined the design
of the interoperation interface:
\begin{itemize}
\item Since both Oz and Alice support higher-order data structures, it is
  most natural to aim for bidirectional interoperability.
\item Futures must be anticipated virtually anywhere in Alice data
  structures.\footnote{Note that we cannot do whole-program analysis due to
  component-based programming.}  Thus, representation analysis techniques
  for SML are not applicable.  This excludes elaborate mappings, only
  leaving us with a very direct approach---with the advantage of becoming
  compositional:  This means that the direct mapping from Alice to Oz will
  immediately imply how to call from Oz to Alice.
\end{itemize}

\noindent Note that if Alice was the host programming system and Oz the
guest language, the situation would be different:  Directly mapping a
dynamically typed language to a statically typed language is impossible
in general---a modelling approach would be mandatory.


%=======================================================================
\section{Mapping Alice to Oz}\label{mapping}
%=======================================================================

We will now present the translation of the Alice core, module, and
component languages to Oz.

%-----------------------------------------------------------------------
\subsection{Core Language}

{\bfseries Primitive data types.}
For many Alice core data types, there are Oz equivalents:  Alice integers
and characters are mapped onto Oz (infinite-precision) integers.  Alice reals
become Oz floats.  Alice strings are represented as Oz byte strings.  Oz
has been extended by a basic data type `word' (with bitwise operations),
provided in C++ using Mozart's foreign function interface.

{\bfseries Records.}
Tuples and records in Alice are very similar to one another:  Tuples are
just syntactic convenience for records whose labels only consist of digits,
which when interpreted as integers are consecutive and start at one.
This maps nicely to Oz:  Oz record features can be either integers or
literals (atoms or names).  Similar to Alice, when a record's features
are consecutive integers starting from one, the record is a tuple.  Our
mapping preserves this similarity by mapping labels consisting only of digits
to integers and all other labels to the corresponding atoms.  In contrast to
Alice, Oz records are equipped with an additional tag, called the \emph{label}:
We chose the label `\#', which is used by the syntactic sugar for mixfix
tuples in Oz.  Thus, the Alice tuple `$(1, 4, 7)$' has the same representation
as the corresponding Oz tuple `1\#4\#7'.  The single exception to this is for
the record that has no fields (that is, the Alice tuple `()' resp.\ record
`\{\}'):  The common idiom in Alice is to use this value as argument of
nullary functions, or as result of functions that only perform a side-effect
and do not return any interesting value.  It is the only value of the
so-called unit type.  Since we want to map idioms, it seems most natural to
represent this as the Oz name \kw{unit} (which incidentally also is an Oz
record without fields).

{\bfseries Constructors.}
Alice constructors are in general mapped to Oz atoms corresponding to the
constructor's identifier, except for exception constructors, which have to
be mapped to Oz names because they are generative.  Some constructors have a
special translation:  For instance, the Alice constructors true and false are
mapped to the Oz names \kw{true} and \kw{false}, to make boolean expressions
interoperate.

{\bfseries Constructed Values.}
Because conceptually all Alice constructors are unary, we have introduced
the notion of \emph{syntactic arity}.  When a constructor is syntactically
declared to take a record (or a tuple) as argument, we consider this
constructor to be $n$-ary, $n$~being the number of fields in the record.
We map constructed values of $n$-ary constructors to Oz records, using
the mapped constructor as label instead of `\#'.  It follows that algebraic
data types are mapped to exactly their usual representation in Oz.
Because we specially translate the constructor `::' to the atom~`$|$',
for instance, the Alice list `1::2::nil' is mapped to the Oz list
`1$|$2$|$nil'.  For syntactic arity to work, we needed to introduce a
similar ML restriction as OCaml~\cite{OCaml} does:  Signature matching is
not allowed to change the syntactic arity of constructors.

{\bfseries State.}
One kind of constructor has a different translation, namely the `ref'
constructor which introduces state.  This is translated to Oz cells, which
support equivalent operations.  In Alice, it can be statically decided
whether a constructor is `ref'.

{\bfseries Functions.}
Like constructors, Alice functions are unary.  We also introduce the concept
of a syntactic arity for functions:  Function definitions use pattern matching
on their arguments.  If any pattern explicitly deconstructs the argument as
a tuple, and no pattern binds the whole tuple to an identifier, then the
function has the tuple's arity.\footnote{Note that we cannot use the function's
type to determine the arity, because this could change the semantics when the
argument is a by-need future:  It would be requested sooner than necessary.}
It is not possible to preserve syntactic arity as for constructors:  This
is why the conversion of calling convention (unary vs. $n$-ary) has to be
performed dynamically.  $n$-ary functions are mapped to $n + 1$-ary Oz
procedures:  The additional argument is used for returning the result.

Note that due to the special handling of the empty tuple (unit), we have
a certain irregularity here:  A nullary Alice function is translated to
a binary Oz procedure expecting the \emph{unit} value as first argument,
instead of a unary Oz procedure taking only one output argument for the
function's result.

{\bfseries Futures.}
Promises, futures and by-needs are implemented as builtins on top of the
functionality of Oz and require no special care.

In terms of representation analysis, arity raising of $n$-ary constructors
and functions are the only optimizations performed.  Note that in the present
case, our primary motivation for this is interoperability, not efficiency.

The type language part of the core language is subject to type erasure.

%-----------------------------------------------------------------------
\subsection{Module Language}

Oz modules are represented as records.  To interoperate smoothly, Alice
structures are thus translated to records with atom features.  Values
are stored under their identifier.  Constructors that take arguments are
represented as two fields:  The constructor's identifier designates the
constructor function, expecting the constructor's argument and returning
the constructed value.  This reflects the fact that a constructor used as
expression is the constructor function. Its identifier prefixed with a
quote (which is not a valid value identifier) designates the constructor
itself and is used to translate pattern matching.

Structures reside in their own namespace, which is distinct from the namespace
for values and constructors.  To avoid capturing, their identifier is prefixed
with a dollar sign, again yielding an invalid Alice identifier.

Functors (functions computing modules from modules) are unary; multiple
arguments are passed as a structure containing structures.  Functors are
translated into binary procedures taking a translated structure as input
and returning a new structure.

Like types, signatures are subject to type erasure.

%-----------------------------------------------------------------------
\subsection{Components}

An Alice component can be considered as a functor, with the extension
that its argument and result may contain nested signatures and functors.%
\footnote{In contrast to SML, Alice supports higher-order functors.}
The component functor explicitly identifies its argument modules by URLs
to instruct the component manager on how it is to be applied.

Because components are inspired by Oz, Alice components and Oz components
are so similar to one another that translation becomes trivial, with some
additional considerations about component signatures.  Alice components
extend Oz components by signatures for imports and export.  To import an
Alice component into Oz, no mechanism besides ignoring these signatures is
necessary.

Alice, being statically typed, can import a dynamically-typed Oz component
only if its export signature is known.  When the Alice compiler encounters
an import directive, it loads the export signature from the referenced
compiled component.  In the case of an Oz component, such an export signature
will be absent.  Instead, the programmer can provide a signature in a file
with the same base name as the referenced component and the special extension
`.asig'.  At run-time, this file is not needed.

We extended the Oz component manager to be aware of component signatures:
When an Alice component imports another Alice component, it will at link-time
verify that their signatures match, as required by the Alice language
definition.  In all other cases, signatures are ignored.

Under this translation, the example components in Fig. \ref{ozcomponents}
and~\ref{alicecomponents} become interchangeable.

%-----------------------------------------------------------------------
\subsection{Libraries}

Alice supports a significant portion of the Standard ML Basis Library~%
\cite{SMLBasis}, making itself largely compatible with SML.  Extensions
adhere to the style of the Standard Basis.  This makes them feel more
natural to SML programmers.


%=======================================================================
\section{Implementation}\label{implementation}
%=======================================================================

Both the implementation of the Alice compiler and the Alice runtime make
reuse of existing Mozart technology.  This is detailed below.

%-----------------------------------------------------------------------
\subsection{Mozart Extensions for Alice}

The Alice runtime is the Mozart System, with minimal extensions.
These are now part of the standard Mozart distribution.

{\bfseries Tupling and Detupling.}
Two instructions have been added to the Mozart Virtual Machine for
run-time calling convention conversions.  The first applies a procedure
with $n > 1$ arguments.  A run-time check tests whether the arity of the
procedure and the call match, in which case the call proceeds normally.
Else, the procedure must be binary (that is, a unary function):  A tuple
is constructed from the first $n - 1$ arguments and passed as a single
argument.  The $n$th argument becomes the second argument and is used
to return the function's result.

The second instruction works the other way round:  If a procedure expecting
more than two arguments is called with only two arguments, the first must
be a tuple which is deconstructed to obtain the actual $n - 1$ input
arguments.  The second argument becomes the $n$th argument and is used
for output.

{\bfseries Typed Components.}
Oz components already carry type fields for the export signature of
components and the expected import signatures of referenced components.
This type representation has been extended to represent Alice signatures,
which are much more expressive than the simple Oz types intended for type
falsification.\footnote{Oz being a dynamically typed language, type
verification is impossible.  Type falsification allows for early error
diagnosis.}

{\bfseries Link-time Type-checking.}
The Mozart component manager and static linker are now parameterized
over a type-checking procedure, to accomodate for Alice link-time type
checking.

%-----------------------------------------------------------------------
\subsection{The Alice Compiler}

The Alice compiler generates Mozart components from Alice sources.
To reuse the bytecode generator from Mozart's Oz compiler, we used
an intermediate representation of the latter as target for the Alice
compiler.  This intermediate representation basically corresponds to
alpha-renamed Core Oz programs in which optimizations have been made
explicit.  This allowed us to directly plug in the existing backend,
performing liveness analysis, instruction selection, and register
allocation.

Thus, the Alice compiler's frontend is implemented in Alice, while its
backend is implemented in  Oz.  The bootstrapped compiler integrates these
two using the Oz-Alice-interoperability described above.  However, for
bootstrapping the compiler, this option is not available.  We use
Standard ML of New Jersey~\cite{SMLofNJ} to compile the frontend, which
is implemented using only the SML subset of Alice.  A dummy backend
writes the abstract syntax tree to an Oz pickle (a persistent representation
of a data structure).  An independent Mozart process, in turn, reads this
pickle and executes the real backend.  The bootstrapped version of Alice does
not require Standard ML of New Jersey to run: it is a pure Mozart application.


%=======================================================================
\section{Discussion}\label{discussion}
%=======================================================================

The motivation behind our work was to provide an implementation of the
programming language Alice, but it can also be argued that with only
minimal extensions to Standard ML's language definition and the Mozart
runtime, we made SML interoperate smoothly with Oz.

Our approach has some limitations however regarding the use of the guest
language Alice from the host language Oz:
\begin{itemize}
\item Alice uses the type-based mechanism of signature abstraction to
  implement abstract data types.  However, due to Oz' dynamic typing,
  Oz programs will be able to break these abstraction barriers:  Alice
  abstract types are concrete as seen from Oz.
\item The Oz compiler does not implement dynamic calling convention
  conversion, which would actually alter the defined semantics of Oz.
  The consequence is that Oz programs have to use the exact arity chosen
  by the Alice compiler when calling Alice functions.  The arity can
  change as the implementation of an Alice function changes, without
  even changing its signature.  Oz programmers can account for this
  situation by performing calling convention conversion at the language
  level, by explicitly testing the procedure's arity.
\item Alice and Oz currently use different exceptions for signaling
  similar error conditions.  In particular, it is impossible to catch
  the exceptions raised by Oz primitives in Alice.
\end{itemize}

\noindent None of these limitations has ever been a serious problem in
practice, although the interoperation interface has already been used
extensively.

In terms of sheer performance, the present implementation of Alice nearly
attains that of Oz, but of course cannot compete with high-performance ML
implementations.  Nevertheless, we believe that in the application domains
in which Oz is used heavily, Alice has a definite advantage over SML.


%--** Summary and Lessons Learned
%--** (relational and functional are not so different)

%=======================================================================
\section{Outlook}\label{outlook}
%=======================================================================

We intend to formalize the Alice extensions in the style of the Definition
of Standard ML.  Work has started on the formalization of futures and laziness.

Already we have several projects running that build on the work in this paper
to exploit the potential of the combination of Oz and Alice.  Oz libraries
are being reshaped as Alice libraries, to give Alice programmers access to
Oz' constraint systems and constraint solving facilities.  Alice already
benefits from the development tools developed for Oz:  A customized version
of the Oz Inspector, intended for interactive value inspection, uses Alice
syntax.  The Oz Explorer, used for visualizing search trees, is available
for Alice constraint problems.  The Oz Profiler collects statistics about
Alice functions.  The usefulness of the Oz debugger for Alice programs
is being investigated.  In the case of a new GTk-based graphical user
interface library, development for Alice has even overtaken that for Oz.

The ongoing work of extending Alice by run-time type representations has
enabled us to enrich the language with dynamically typed packages and
type-safe pickling.  Also, the first type-safe distributed applications
are now running.



%=======================================================================
\ack
%=======================================================================

Many thanks go to Ulrike Becker-Kornstaedt, Thorsten Brunklaus, Tobias
M�ller, and Christian Schulte for their comments on a previous version
of this paper.  For the numerous discussions regarding the details of
the translation scheme, thanks go to Andreas Rossberg, implementor of
the Alice compiler frontend.


\bibliographystyle{entcs}
\bibliography{babel01}

\end{document}
